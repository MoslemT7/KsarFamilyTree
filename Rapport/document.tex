\documentclass[a4paper,12pt]{report}

\usepackage{graphicx}
\usepackage{fancyhdr}
\usepackage{geometry}
\geometry{a4paper, margin=1in}

\begin{document}
	
	% Page de couverture
	\begin{titlepage}
		\centering
		\vspace*{1cm}
		
		% Logo de l'université
		\begin{figure}[h]
			\centering
			\includegraphics[width=0.3\textwidth]{ISIMM.png} % Remplacez "logo_universite.png" par le nom du fichier du logo de votre université
		\end{figure}
		
		\vspace{1cm}
			\Large
		Université Institue Supérieur d'Informatiqu et de Mathématiques de Monastir \\
		Département d'Informatique \\
		Année académique 2024/2025
		
		\vspace{0.8cm}
		
		\Large
		[Date de soumission]
		
		\Huge
		\textbf{Ksar Family Tree}
		
		\vspace{0.5cm}
		\LARGE
		Projet de Fin d'Études
		
		\vspace{1.5cm}
		
		\textbf{Réalisé par :} \\
		\Large Moslem Teyeb
		
		\vspace{1cm}
		
		\textbf{Encadrant :} \\
		\Large Nom de l'Encadrant
		
		\vfill
		
	
	
	\end{titlepage}
	\tableofcontents
	\chapter*{Remerciements}
	
	Je tiens à exprimer ma sincère gratitude envers toutes les personnes qui ont contribué, de près ou de loin, à la réalisation de ce projet.
	
	Je remercie tout particulièrement mon encadrant, [Nom de l'encadrant], pour son soutien constant, ses précieux conseils, et ses orientations qui m'ont permis d'orienter ce projet de manière efficace et pertinente. Son expertise et sa disponibilité ont été essentielles à la bonne avancée de ce travail.
	
	Je remercie également ma famille, en particulier [nom des membres], pour leur soutien moral et leur compréhension durant les moments difficiles du projet. Leur encouragement et leur patience ont été des moteurs pour surmonter les défis rencontrés.
	
	Enfin, je souhaite remercier toutes les personnes qui m’ont apporté leur aide, que ce soit par des échanges, des critiques constructives ou des conseils, et qui ont permis d'enrichir ce projet à travers leurs contributions.
	
	\chapter*{Résumé}
	
	Le projet \textit{Ksar Family Tree} consiste à la création d'une plateforme interactive dédiée à la gestion et à la visualisation d'un arbre généalogique pour la famille de la ville de Ksar Ouled Boubaker. Ce projet vise à digitaliser les informations familiales et à permettre une consultation facile et intuitive des relations au sein de la famille. Le système s’appuie sur une architecture moderne à trois couches (3-tiers) pour assurer une bonne performance, une sécurité renforcée, et une accessibilité à grande échelle.
	
	L'objectif principal du projet est de permettre à la communauté de suivre l’évolution de leur arbre généalogique, de découvrir des liens de parenté, et de préserver les informations historiques de la famille. Pour ce faire, une interface web est développée, permettant de naviguer facilement entre les différentes générations et d'ajouter de nouvelles données.
	
	La méthodologie utilisée repose sur des technologies robustes telles que PostgreSQL pour la gestion des données, et des frameworks comme React.js et Node.js pour le développement des interfaces et de l'API. Les résultats attendus incluent une application fonctionnelle, intuitive et sécurisée, capable de traiter et de visualiser des milliers de données familiales avec une interface conviviale.
	
	\newpage
	\chapter{Introduction}
	\section{Présentation du problème et motivations}
	Le projet \textit{Ksar Family Tree} vise à répondre à un besoin crucial de gestion et de préservation des liens familiaux dans la ville de Ksar Ouled Boubaker. La famille étant au cœur de la culture locale, les informations familiales sont souvent dispersées et difficiles d’accès. L’objectif est de créer une plateforme numérique permettant de visualiser, gérer et suivre l’évolution de l’arbre généalogique de la communauté, tout en préservant l’aspect historique des relations familiales.
	
	Pourquoi ce projet a-t-il été choisi ?
	Ce projet a été choisi pour des raisons personnelles et professionnelles. D’un côté, il permet de contribuer à la gestion des données familiales dans ma communauté et d’offrir une solution adaptée aux besoins locaux. D’un autre côté, il me permet de développer mes compétences en développement logiciel, en particulier en React.js et Node.js, tout en appliquant des principes d’architecture logicielle à grande échelle. Ce projet est ainsi une opportunité de mêler engagement personnel et développement professionnel.
	\section{Objectifs du Project:}
	\subsection{Objectifs Principaux:}
	Le projet Ksar Family Tree a pour objectif principal de créer une plateforme numérique permettant aux habitants de Ksar Ouled Boubaker de visualiser, gérer et suivre l’évolution de leurs liens familiaux à travers un arbre généalogique interactif et accessible. L’objectif est de centraliser et de numériser les données familiales dispersées, facilitant ainsi la consultation et la compréhension des relations au sein de la communauté.
	\subsection{Finalité du Project:}
	La finalité du projet est de fournir à la communauté de Ksar Ouled Boubaker un outil simple et intuitif pour préserver son patrimoine familial et culturel, tout en facilitant les échanges entre les membres de la famille.
	
	\chapter{Analyse des Besoins:}
	
\end{document}
